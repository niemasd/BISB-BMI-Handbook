\chapter{Mission Statement}
We are witnessing the birth of a new era in biology and medicine. The confluence of unprecedented measuring capabilities and computational power has dramatically changed the questions that may be addressed in the biological and biomedical sciences and promises to empower clinical practice in fundamental ways.

On the one hand, recent and novel technologies produce biological data sets of ever-increasing resolution that reveal not only genomic sequences, but also RNA and protein abundances, their interactions with each other, their sub-cellular localization, and the identity and abundance of other biological molecules. This requires the development and application of sophisticated computational methods, encompassed by the field of Bioinformatics.

On the other hand, biomedical research has risen to the challenge of understanding the integrated functions of thousands of genes. Physical and functional interaction networks chart connectivities, reveal functional modules, and provide clues on the functioning of specific genes. Using mathematical models of the stochastic and dynamical events of biology reveals fundamental design principles and allow for virtual experimentation.  This is a focus of the field of Systems Biology.

In addition, rapidly increasing capabilities of rapid molecular and genomic analyses in the clinic promise to transform medical practice in unprecedented ways.  The ability to cross-query data and knowledge bases provides opportunities and challenges to computational sciences interfacing with medicine to produce support systems for data management, text and language processing, privacy, clinical decision support, and data mining for knowledge discovery. These define the goals of Biomedical Informatics.

Addressing these challenges requires an interdisciplinary research structure dedicated to developing intellectual and human capacity in Bioinformatics and Systems Biology (BISB) and Biomedical Informatics (BMI). As such, there is an enormous need for trained professionals who are experts in biology, biomedicine, and computing. To address this need, the Bioinformatics Graduate Program at the University of California, San Diego, was founded in 2001 by Professor Shankar Subramaniam. It now includes five Schools and Divisions on the UCSD campus: The Jacobs School of Engineering (Bioengineering, Computer Science and Engineering, and Nanoengineering), The Division of Biological Sciences (Molecular Biology, Cell and Developmental Biology, Neurobiology, Ecology/Behavior/Evolution), the Division of Physical Sciences (Chemistry \& Biochemistry, Physics, and Mathematics), the School of Medicine, and the Skaggs School of Pharmacy and Pharmaceutical Sciences. The Graduate Program is supported by the respective schools, divisions, and departments as well as by a substantial NIH Training Grant and over fifty associated faculty.