\chapter{Program Administration}

\section{Steering Committee}
The policies and guidelines that govern the program are set by the Steering Committee. This committee oversees a broad range of issues affecting the program including Admissions, budget spending, and new faculty applications. This committee is composed of the chairs of the various faculty-student committees, faculty representatives of numerous departments, as well as three student representatives.

\section{Program Director}
The Director of the program is appointed for a renewable 3-year term by the Steering Committee, based on input from the program faculty, and may have their primary appointment in any of the departments that make up the faculty group. The Program Director is a full-time member of the program faculty and is responsible for the overall administration of the PhD Program. The Program Director serves as chair of the Steering Committee.

\section{Program Coordinator}
The Graduate Program/Affairs Coordinator serves all students and affiliated faculty members in academic, financial, and administrative matters relating to the BISB Graduate Program. The Graduate Coordinator is the primary contact for students on most-program related matters. This includes, but is not limited to, advising on coursework, enrollment, programmatic and university milestones, funding, fellowship support, and program events.

\section{Division of Graduate Education and Post Doctoral Affairs (GEPA)}
The Division of Graduate Education and Post Doctoral Affairs (GEPA), also known as Graduate Division or Grad Division, administers all graduate programs at UCSD. This office is responsible for all aspects of graduate programs, from admissions and administration of fellowships, to Senate Exams, and Defense Dissertations. Think of GEPA as central, campus administration, and BISB Program Administration as focused on the intricacies of the program.

\section{Additional Committees}

\subsection{Curriculum Committee}
The Curriculum Committee is a faculty-only committee that advises the Program Director on student petitions. Course petitions allow flexibility for students and acknowledge individual student needs and experiences. Common petitions include substituting and waiving electives. Petitions should be sent by email to the Graduate Coordinator for initial review; they are then relayed to the Curriculum Committee.

\subsection{Outreach and Student Affairs Committee}
Another faculty-only committee, the Outreach and Student Affairs Committee is responsible for keeping the course catalog and website up to date; keeping records of student publications and other records relevant to the Bioinformatics T32 Training grant; rotation reports, including faculty research reports; alumni tracking; and developing extramural links of the BISB Program.

\subsection{Colloquium Committee}
Jointly run with members from the Genetics Training Program, the Colloquium Committee handles the \href{http://genomic.weebly.com/}{Genetics, Bioinformatics, and Systems Biology Colloquium (GBSBC)}, a weekly seminar series that runs during the academic year. Attending the GBSBC is a requirement of the seminar course \textbf{BNFO 281}. Members of the Colloquium Committee are responsible for organizing all aspects of the GBSBC, including securing speakers from around the world.

\subsection{Diversity, Equity, and Inclusion (DEI) Committee}
The Diversity, Equity, and Inclusion (DEI) Committee is a group of dedicated students, faculty, and staff who work on promoting and enhancing diversity and inclusion efforts within BISB and beyond in the field. Recent efforts have included organizing DEI Coffee Hours where students can connect with faculty, and creating and circulating the BISB social climate survey.

\subsection{Fellowship Committee}
The Fellowship Committee is a faculty-only Committee that is responsible for nominating students for fellowships; finding and publicizing fellowships of interest to BISB students; and holding seminar to help students apply for fellowships.

\subsection{Admissions Committee}
The Admissions Committee is a large, faculty-only Committee whose members vary by admissions cycle. Faculty in this Committee are responsible for reviewing applications to the BISB/BMI Programs, conducting interviews, and nominating candidates for admission, all while abiding by a holistic review process. The Admissions Committee is dedicated to DEI efforts.